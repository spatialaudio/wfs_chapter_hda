Wir danken Franz Zotter herzlich für die Entstehungsdynamik des
Open Source \& Open Educational Resources Projektes \cite{ZotterSchultz2020_White} und seiner
damit verbunden initialen Niederschrift des hier wiedergegebenen
Herleitungsformalismus, sowie für zahlreiche Hinweise und Korrekturvorschläge
für dieses Kapitel.
%
Ein herzlicher Dank geht auch an Annika Neidhardt für ausgiebiges Korrekturlesen und
Verbesserungsvorschläge.
%
Wir danken Matthias Geier für den Hinweis auf den Artikel \cite{Fletcher1934},
der zur Einordnung der WFS-Ingenieursutopie nicht fehlen sollte.
