% wfs_chapter_hda
% - git repository https://github.com/spatialaudio/wfs_chapter_hda
% - drafts for the chapters (english, german) on **Wave Field Synthesis** for
% Stefan Weinzierl (ed.): *Handbuch der Audiotechnik*, 2nd ed., Springer,
% https://link.springer.com/referencework/10.1007/978-3-662-60357-4
% - text and graphics under CC BY 4.0 license https://creativecommons.org/licenses/by/4.0/
% - source code under MIT license https://opensource.org/licenses/MIT
% - Springer has copyright to the final english / german chapters and their layouts
% - we might also find https://git.iem.at/zotter/wfs-basics useful
% - we use violine image from https://upload.wikimedia.org/wikipedia/commons/thumb/f/f1/Violin.svg/2048px-Violin.svg.png to create picture `python/violin_wfs.png`

% Authors:
% - Frank Schultz, https://orcid.org/0000-0002-3010-0294, https://github.com/fs446
% - Nara Hahn, https://orcid.org/0000-0003-3564-5864, https://github.com/narahahn
% - Sascha Spors, https://orcid.org/0000-0001-7225-9992, https://github.com/spors
%
Wellenfeldsynthese (WFS) ist ein räumliches Wiedergabeverfahren, mit dem
räumlich und zeitlich gestaffelte Wellenfronten mittels kontrollierter
Interferenz synthetisiert werden.
%
Dazu benötigt es Lautsprecherarrays mit sehr dichtem Lautsprecherabstand und
individueller Signalverarbeitung für die Lautsprecher.
%
Im Gegensatz zu kanalbasierten Wiedergabeverfahren werden bei der WFS die
Lautsprechersignale errechnet wofür Messdaten oder Audio-Objekte, und deren
räumlich-zeitliche Parametrik einfließen.
%
Anwendung findet die WFS bei lautsprecherbasierter Auralisation,
Nachhallsynthese in Räumen, 3D-Beschallungskonzepten,
audiologischer Grundlagenforschung und
spatialisierter Audiokunst.
%

Das Kapitel diskutiert die akustischen Grundlagen der WFS, enthält eine
kompakte, anschauliche Herleitung der WFS einer virtuellen Punktquelle und
arbeitet die wichtigsten Eigenschaften der mit WFS synthetisierten Schallfelder
heraus.
%
Abschließend werden praxisrelevante WFS-Modifikationen und -Werkzeuge dargelegt.
